\section{Conclusion}
\label{sec.conclusion}

Isolating untrusted user applications from the underlying kernel is desirable, in order to protect the privileged code and 
avoid the exploitation of bugs. However, there has yet to be a standard method that can shed lights on how to effectively
isolate the user space applications from the kernel space without losing desired functionality.

We propose a new metric that quantitatively measures and evaluates the kernel code being executed when running
user applications. The kernel trace profiling data obtained through our metric provides useful insights into different 
features of the kernel and can be used to devise new designs that aims at specific requirements, such as providing 
strong security to the system. 

We implemented a sandbox system, called Lind, which is based upon new design using our metric. 
Our system Lind, securely reconstruct complex yet essential OS functionality inside a dual-layer sandbox. 
The sandbox itself is designed to have a minimized trusted computing base (TCB) and only interact with the kernel
in a minimal and safe way. 

Our evaluation results have shown that Lind is least likely to trigger historically reported kernel bugs, compared against
other systems built without using our metric, such as VirtualBox and Graphene. The successful implementation of Lind
suggests that new designs using our metric will lead to more secure systems. Moreover, Our metric is likely to encourage
new designs that can create systems that suit their goals and requirements better. 