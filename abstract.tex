\begin{abstract}

An Operating System (OS) kernel is a trusted part of modern computer systems, 
yet despite substantial effort, kernels still contain bugs and are vulnerable to many types of 
attacks. In this paper, we explained why existing techniques failed to protect the kernel effectively. 
Moreover, to provide a solution, we devised a new metric that quantitatively measures and 
evaluates different lines of code in the kernel based upon how often these lines 
are executed by running user space applications. Using our metric, we can obtain 
the kernel trace profiling data, which is critical to designing new systems to meet their purposes
and requirements more effectively. 

For example, we used our metric to come up with a new design that aims at providing strong
isolation between the kernel space and the user space and therefore protecting the kernel 
from being exploited. Using this new design, we implemented a sandbox called Lind, which minimized
the trust placed in risky kernel code by containing risky code that is needed to support legacy 
programs within a sandbox. We compared kernel traces generated by running user applications 
under Lind and other systems without using our design, such as VirtualBox and Graphene. 
Using historical kernel bug reports, we verified that Lind is least likely to trigger kernel bugs. 
Our evaluation results demonstrate that designs using our metric, such as our Lind prototype, 
will lead to more secure systems.

\end{abstract}