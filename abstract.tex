\begin{abstract}

An Operating System (OS) kernel is a trusted part of modern computer systems. \linda{new sentence} Yet, despite substantial effort \linda{what kind of effort, or by who/what?}, kernels still contain bugs and \sout{are} remain vulnerable to many types of attacks. In this paper, we \sout {explained} explain why existing techniques fail to effectively protect the kernel \sout{effectively}. \linda {watch for incorrect tenses - especially past.I will fix them.}
\sout{Moreover, to provide a solution,} We devised a new metric that quantitatively measures and evaluates different lines of code in the kernel based upon how often these lines are executed by running user space applications. Using our metric, we can obtain \sout{the} kernel trace profiling data, which is critical to designing new systems \sout{to meet their purposes and requirements more effectively.} 

For example, we used our metric to come up with a new design \linda {design of what? the reader is likely to assume you designed a system or tool; so maybe you want to say design concept or something similar} that aims at providing strong isolation between the kernel space and the user space and therefore protects the kernel from being exploited. Using this new design, we implemented a sandbox (that we call Lind), which minimizes the trust placed in risky kernel code by containing needed but risky code that \sout {is needed to} supports legacy programs \sout{within a sandbox}. We generated and compared kernel traces \sout{generated} by running user applications under Lind and in other \linda{add: non-Lind} systems, such as VirtualBox and Graphene. Using historical kernel bug reports, we verified that Lind is the system least likely to trigger kernel bugs. Our evaluation results demonstrate that designs using our metric, such as our Lind prototype, will lead to more secure systems. \linda{I'm not sure you want to use the word 'system' to describe Lind and others. Are they environments? sandboxes?...}

\end{abstract}