\section{A New Design for Building Secure Systems}
\label{sec.design}

In this section, we introduce a new design for building secure systems. More specifically, our goal is to 
build secure systems that can mitigate the problem of kernel exploitation. As we have discussed in 
the motivation section \S{2} before, one key reason that many previous work failed to prevent kernel exploitation
effectively is that there is not enough knowledge about which portions of the kernel can be safely exposed
to the user applications. And there is no standard method to acquire this knowledge. The metric we introduced
in \S{3} is proposed to serve the purpose of having a standard method to obtain better understanding of the 
kernel, and know which portions of the kernel can be safely exposed. We shall then use our new metric to 
gain insights into the kernel and use those new insights to provide guidelines for our new design. 

\subsection{Threat Model}
The goal of our new design is to build secure systems that can effectively prevent the problem
of kernel exploitation. We first discuss the threat model to our new design.

In our threat model, threats refer to any behavior that may cause potential harm to the system, 
which may be triggered by malicious code or bugs in non-malicious programs.

In order to fully execute their functions, applications need to have access to a set of privileges provided by 
the operating system, usually exposed as system calls. The primary security goal of a secure system is to 
restrict a program to some subset of privileges, usually by exposing a set of functions that mediate 
access to the underlying operating system privileges. Threats occur when applications obtain access to 
privileges that were not intentionally exposed by the system, thus escaping the protection 
intended by the secure system \cite{Repy:10}.

To pose threats to a secure system, we assume that applications may use multiple threads to modify visible 
state or issue concurrent requests which may trigger a race condition. Our goal is to prevent bugs in 
the code from allowing an user program to escape the secure system to access the kernel code, and 
possibly trigger kernel bugs.

\subsection{Design Guidelines}
Guidelines for new designs aiming at building secure systems are drawn from our metric. 
More specifically, we construct our design guidelines from our metric by the following steps:

\begin{enumerate}
\item Decide which programs will the designed system support.
\item Run those programs from 1) and use our metric to obtain kernel profiling data.
\item Based on the kernel profiling data from 2), decide which portions of the kernel has been used.
\item Check against kernel bug reports to decide which portions of the kernel can be safely exposed.
\end{enumerate}

Using the steps described above, we constructed the guidelines for our new design; 
 
\begin{table}[ht]
\centering
\begin{tabular}{|l|c|}
  \hline
  Kernel Path & Safe to Expose? \\
  \hline \hline
  arch/x86/include/asm & yes \\
  \hline
  arch/x86/lib & yes \\
  \hline
  block & yes \\
  \hline
  kernel & yes \\
  \hline
  kernel/cpu & yes \\
  \hline
  kernel/gcov & yes \\
  \hline
  kernel/locking & no \\
  \hline
  kernel/sched & yes \\
  \hline
  net/unix & no \\
  \hline
\end{tabular}
\caption {Design Guidelines}
\label{table:design_guidelines}
\end{table}

\subsection{Design Choices}
The design guidelines from last section is an useful indication of which portions of the kernel code can be
safely exposed. When we create the design, we make specific design choices base on the guidelines.

\begin{enumerate}
\item Where to place the system: In the user space. 
\item What is the proper interface: decide a list of system calls to support.

\begin{table}[ht]
\centering
\begin{tabular}{|l|c|}
  \hline
  System Call & How to Support? \\
  \hline \hline
  access & direct support \\
  \hline
  unlink & safely reimplement \\
  \hline
  open & safely reimplement \\
  \hline
  close & safely reimplement \\
  \hline
  read & safely reimplement \\
  \hline
  write & safely reimplement \\
  \hline
  lseek & safely reimplement \\
  \hline
  fstatfs& safely reimplement \\
  \hline
  socket & safely reimplement \\
  \hline
  bind & safely reimplement \\
  \hline
  fork & no support \\
  \hline
\end{tabular}
\caption {Design Choices}
\label{table:design_choices}
\end{table}

\end{enumerate}
